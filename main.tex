\documentclass[11pt]{article}
\usepackage{lindrew}

\title{Notas de Homotopía I}
\author{Luis Zegarra \\ Profesor: Bruno Vallette \\ IMJ-PRG \\ Correo electronico: luis.zegarrab@pucp.edu.pe}

\date{\today}

\begin{document} 

\maketitle

La meta del curso será presentar dos teorías concretas de homotopía. Empezaremos con la homotopía clásica de espacios topológicos (grupos de homotopía, complejos celulares, teorema de Whitehead, teorema de Hurewicz y fibraciones), y luego estudiaremos la homotopía de espacios simpliciales (definiciones, categoría de símplices, adjunción y objetos co-simpliciales, ejemplos, fibraciones, complejos de Kan y homotopía simplicial). Las nociones serán introducidas y discutidas con miras hacia la definición de $\infty$-categorías.

\tableofcontents

\newpage

\section{Clase del 04 de Noviembre del 2024}

% lecture_01.tex
% Lecture 01: Introduction
% This file is intended to be included in the main document via \input


En este curso discutiremos ideas y motivaciones. Para los detalles técnicos, nos remitiremos a las notas de clase o a los libros de texto.

Hacer preguntas es fundamental. Hilbert solía decir que "la pregunta es más importante que la respuesta". Nadie comprende algo a la primera; es a través de la indagación y la discusión que alcanzamos una comprensión más profunda.

\subsection{Motivación}

Comencemos con una perspectiva histórica.

Queremos entender el problema de la deformación. El primer problema de deformación surge en el contexto de los \textbf{espacios topológicos}. Nuestros ejemplos favoritos de espacios topológicos son las \textbf{esferas}, los \textbf{toros} y los \textbf{espacios euclidianos}. Queremos saber cuándo dos espacios topológicos son ``el mismo''. Pero, ¿qué significa ``el mismo''?

Existen varias opciones:

\begin{itemize}
    \item \textbf{Espacios homeomorfos:} Dos espacios topológicos son homeomorfos si existe una función continua, biyectiva y con inversa continua entre ellos. Esto significa que son ``el mismo'' desde el punto de vista topológico.
\end{itemize}

Por ejemplo, un círculo no es equivalente a un toro sólido, ya que si removemos dos puntos de cada uno, el círculo permanece conexo, mientras que el toro no.

Sin embargo, esta noción de ``similitud'' es muy estricta. Una mente inocente podría pensar que el toro sólido es el mismo espacio que el círculo, pues si comprimimos el toro sólido, podemos deformarlo hasta obtener un círculo.

Esta última idea es interesante. Muchas propiedades relevantes de los espacios topológicos se preservan bajo deformaciones continuas, como la conectividad y el número de agujeros. Por ello, nos interesa establecer una noción de equivalencia más débil que la de homeomorfismo, pero que refleje la idea intuitiva de deformación continua. Esto nos lleva a la siguiente noción:

\begin{itemize}
    \item \textbf{Equivalencia homotópica:} Dos espacios topológicos son homotópicamente equivalentes si se pueden deformar uno en el otro mediante una serie de deformaciones continuas. Así, aunque no sean homeomorfos, comparten ciertas propiedades topológicas esenciales.
\end{itemize}

La manera formal de abordar esto es dejar de lado los espacios y concentrarnos en las funciones continuas entre ellos. Establecemos una relación de equivalencia entre funciones continuas que nos permite identificar cuándo dos espacios son homotópicamente equivalentes.

Este cambio de perspectiva, de puntos (en el caso topológico) a funciones (en el caso homotópico), nos permite utilizar herramientas algebraicas para estudiar propiedades topológicas. Los espacios no suelen tener buenas propiedades algebraicas, pero las funciones sí, y estas provienen del mundo de las categorías.

Esta noción de equivalencia funciona bien para varios tipos de espacios, como los CW-complejos, variedades, etc. Una vez establecida la noción de equivalencia, surge la primera pregunta:

\begin{pregunta}
    ¿Podemos clasificar los espacios topológicos módulo equivalencia homotópica?
\end{pregunta}

La forma canónica de abordar esta pregunta es mediante la construcción de invariantes algebraicos. Es decir, a partir de un espacio $X$ construimos un objeto algebraico $G(X)$ que sea estable bajo equivalencias homotópicas. Si $X$ es homotópicamente equivalente a $Y$, entonces $G(X)$ es isomorfo a $G(Y)$.

Por ejemplo, para distinguir $\mathbb{R}^3$ de $S^1$, podemos considerar el grupo fundamental $\pi_1(X)$, que es un invariante homotópico. En este caso, $\pi_1(\mathbb{R}^3) = 0$ y $\pi_1(S^1) = \mathbb{Z}$, lo que nos permite distinguir ambos espacios.

\begin{pregunta}
    ¿Podemos encontrar invariantes totales? Es decir, ¿existen invariantes tales que, si coinciden para dos espacios distintos, entonces los espacios son homotópicamente equivalentes?
\end{pregunta}

Así comienza esta historia, a principios del siglo XX, con el trabajo de Henri Poincaré. Poincaré trabajó primero con variedades y obtuvo los invariantes dados por los \textbf{grupos de homología y cohomología}. Lamentablemente, estos grupos no eran invariantes totales. Luego surgieron los \textbf{grupos de homotopía}, que tampoco lo son. Sin embargo, aparecieron nuevos invariantes: estructuras algebraicas superiores sobre estos grupos, como el \textbf{anillo de cohomología}. Pero incluso estas estructuras más sofisticadas fallaban en ser invariantes totales.

No fue hasta finales del siglo XX que se consideraron las \textbf{estructuras algebraicas infinitas}, que son, en esencia, álgebras homotópicamente conmutativas actuando sobre el anillo de cohomología. Finalmente, estas estructuras sí son invariantes totales y nos permiten clasificar los espacios topológicos módulo equivalencia homotópica.

\begin{pregunta}
    ¿Significa esto que el trabajo ha terminado? ¿Ya no hay nada por hacer?
\end{pregunta}

¡En absoluto! Aún hay muchas cuestiones abiertas. Por ejemplo, ni siquiera podemos calcular los grupos de homotopía de las esferas en todos los casos.

Sin embargo, en cierto sentido, hemos ``linealizado'' la teoría, pues esencialmente hemos reducido el problema a uno de un álgebra actuando sobre un módulo.

\begin{pregunta}
    Esto último parece muy difícil. ¿Deberíamos rendirnos?
\end{pregunta}

Eres libre de rendirte si lo deseas, pero yo no lo haré. ¿Por qué? Porque la buena matemática surge de buenas preguntas. La pregunta que surgió de toda esta discusión en busca de invariantes totales es natural:

\begin{center}
    \textbf{¿Cómo comparamos distintos tipos de invariantes?}
\end{center}

De aquí nace la necesidad de desarrollar herramientas que nos permitan comparar estos invariantes de manera efectiva. Esto desembocó en el desarrollo de la \textbf{Teoría de Categorías}. La teoría de categorías nos proporciona un marco unificado para entender y comparar diferentes tipos de invariantes a través de la noción de \textbf{functores} y \textbf{transformaciones naturales}.

La teoría de categorías encapsula la noción de functorialidad. Los matemáticos no necesitaban definiciones formales para intuir la functorialidad, pero no fue hasta que surgió la necesidad de comparar invariantes como $\pi_1(X)$ y $H^*(X)$ que realmente se requirió un lenguaje más adecuado: el de ``flechas entre flechas''.

Este nuevo lenguaje nos permitió entender mejor las relaciones entre los invariantes y también unificar el lenguaje utilizado hasta entonces. Además, permitió ver relaciones y similitudes en otras teorías matemáticas. En 1967, Quillen estudió la teoría de homotopía de espacios, donde aparecían las nociones de fibraciones, cofibraciones, categoría homotópica, equivalencias homotópicas, equivalencias débiles, etc.; también estudió conjuntos simpliciales, donde aparecían nociones análogas como las fibraciones, cofibraciones, grupos de homotopía, equivalencias débiles; y también estudió complejos de cadenas, donde también surgieron las nociones de fibraciones, cofibraciones, etc. A partir de estas observaciones, Quillen desarrolló la teoría de modelos, que proporciona un marco para estudiar estas estructuras de manera sistemática.

El objetivo de este curso es estudiar los dos primeros ejemplos: \textbf{teoría de homotopía de espacios} y \textbf{conjuntos simpliciales}. Esto complementa el estudio de la teoría de complejos de cadenas vista anteriormente y nos deja listos para empezar el estudio de categorías de modelos en el futuro. No vamos a ser exhaustivos. La idea es que al final del curso, consigamos una buena comprensión de las ideas y motivaciones detrás de la teoría de homotopía, así como de los conceptos básicos necesarios para abordar la teoría de modelos y las $\infty$-categorías.

\begin{pregunta}
    ¿Por qué un geómetra simpléctico o un algebrista de representaciones debería interesarse en este curso?
\end{pregunta}

En los últimos años, hemos visto que las categorías infinitas han proporcionado un contexto fundamental para formular y abordar problemas tanto en geometría como en álgebra. Por ejemplo, han sido esenciales en el desarrollo del programa de Langlands y en la geometría algebraica moderna.

\subsection{Nociones básicas de homotopía y categorías}

Durante las primeras tres semanas nos centraremos en la teoría homotópica de espacios topológicos. Esto incluirá una revisión de los conceptos fundamentales de homotopía, así como una introducción a las categorías y funtores.

\subsubsection{Categorías Homotópicas}

Vamos a trabajar con la categoría $\mathbf{Top}$. Esta categoría tiene como objetos los espacios topológicos y como morfismos las funciones continuas entre ellos. Dentro de esta categoría, la noción de isomorfismo coincide con la de homeomorfismo.

Los ejemplos fundamentales para la teoría de homotopía son:

\begin{itemize}
    \item El espacio vectorial $\mathbb{R}^n$.
    \item El $n$-disco $\mathbb{D}^n = \left\{ x \in \mathbb{R}^{n} \mid \|x\| \leq 1 \right\}$.
    \item Las esferas $S^n = \partial \mathbb{D}^{n+1} = \left\{ x \in \mathbb{R}^{n+1} \mid \|x\| = 1 \right\}$.
    \item El intervalo $I = [0, 1] \subset \mathbb{R}$. En cierto sentido, es gracias a este espacio que tenemos teoría de homotopía en espacios.
    \item Las cajas $I^n=[0, 1]^n \subset \mathbb{R}^n$.
    \item Los símplices $\Delta^n = \left\{ (x_0, \ldots, x_n) \in \mathbb{R}^{n+1} \mid x_0 + \ldots + x_n = 1,\, x_i \geq 0 \right\}$ y sus bordes $\partial \Delta^n$.
    \item Los espacios proyectivos $\mathbb{R}\mathbb{P}^n$ y $\mathbb{C}\mathbb{P}^n$.
\end{itemize}

Recordemos formalmente la definición de homotopía. Debemos dar una definición que capture la idea de deformación continua de un mapa $f$ a un mapa $g$.

Intuitivamente, parece una buena idea considerar a una homotopía entre $f\colon X \to Y$ y $g\colon X \to Y$ como un camino continuo 
\[
    H : I \to C(X, Y)
\]
donde $H_0=f$ y $H_1=g$. 

Sin embargo, esta definición exige una topología en $C(X,Y)$, la cual no es trivial de definir. Por ello, vamos a considerar una definición más sencilla. Luego exploraremos más esta definición.

Por ahora, una \vocab{homotopía} entre $f\colon X \to Y$ y $g\colon X \to Y$ será simplemente un mapa continuo 
\[
    H: I \times X \to Y
\]
tal que $H(0,x) = f(x)$ y $H(1,x) = g(x)$ para todo $x \in X$. 

Cuando dos mapas $f$ y $g$ en $C(X,Y)$ están relacionados por alguna homotopía, diremos que son \vocab{homotópicos}. 

Empecemos con la siguiente observación:  

\begin{proposicion}[Homotopía es una relación de equivalencia]
    Dados $X$ y $Y$ espacios topológicos, la relación de homotopía es una relación de equivalencia en $C(X,Y)$. 
\end{proposicion}

Introduzcamos un poco de notación. 

\begin{itemize}
    \item Al conjunto de clases de equivalencia de $C(X,Y)$ bajo la relación de homotopía lo denotaremos por $[X,Y]$.
    \item Si dos mapas $f,g\colon X \to Y$ son homotópicos, escribiremos $f \simeq g$ y denotaremos a la clase de equivalencia por $[f]$.
    \item Si un mapa $f\colon X \to Y$ es homotópico a una constante, diremos que es \vocab{homotópicamente nulo}. En cierto sentido, $[f]$ es un cero en $[X,Y]$.
\end{itemize}

El siguiente es un paso filosófico clave. Si $f$ se puede deformar en $g$, vamos a entenderlos como el mismo mapa. Gracias a esta nueva idea, podemos acceder a una nueva categoría, \textbf{HoTop}. Esta categoría debería olvidar la rigidez de Top y permitirnos trabajar con los espacios topológicos de una manera más flexible. 

Luego, HoTop debe tener como objetos a los espacios topológicos y, como morfismos entre $X$ e $Y$, a las clases de homotopía $[X,Y]$. Además, debe venir acompañada de un functor que ``olvide la rigidez'':
\[
    \pi : \mathbf{Top} \to \mathbf{HoTop}
\]
definido por $f \mapsto [f]$.

Para que esto funcione, debemos definir la composición de morfismos en HoTop, y solo hay una forma de definirla:

\[
      \circ : [Y,Z] \times [X,Y] \to [X,Z] \qquad ([g],[f]) \mapsto [g \circ f]
\]

Por supuesto, debemos verificar que esta composición está bien definida. Una vez lo hayamos logrado, es claro que tendremos una categoría y su respectivo functor asociado.

\begin{proposicion}[HoTop como una categoría cociente de Top]
    La composición está bien definida en HoTop. En particular, HoTop es una categoría y $\pi$ es un functor.
\end{proposicion}

Ahora que tenemos una nueva categoría \textbf{HoTop} y un functor $\pi$, podemos hacer algunas preguntas importantes.

\begin{pregunta*}
    ¿Cuál es la noción de isomorfismo en HoTop? ¿Cuándo dos espacios son isomorfos en HoTop?
\end{pregunta*}

Abriendo definiciones, diremos que dos espacios topológicos $X$ y $Y$ son \vocab{homotópicamente equivalentes} (isomorfos en HoTop) si existen mapas continuos $f\colon X \to Y$ y $g\colon Y \to X$ tales que $g \circ f$ es homotópico a la identidad en $X$ y $f \circ g$ es homotópico a la identidad en $Y$. 

En este caso, diremos que $f$ y $g$ son \vocab{equivalencias homotópicas}. Sus clases de homotopía son exactamente los isomorfismos en HoTop.

\begin{nota}
    En topología algebraica encontraremos que para comparar espacios u objetos no usamos propiedades intrínsecas, sino que buscamos morfismos entre los objetos que nos permitan compararlos de manera más precisa. En este caso, las equivalencias homotópicas nos dan nociones concretas de comparación que significan equivalencias en HoTop.
\end{nota}

Tenemos algunos ejemplos de espacios homotópicamente equivalentes:

\begin{itemize}
    \item Todo espacio $X$ es homotópicamente equivalente a $X \times I$.
    \item $\mathbb{R}^n$ es homotópicamente equivalente a $\{pt\}$. Este tipo de espacios, los homotópicamente equivalentes a un punto, son llamados \vocab{contráctiles} y son especialmente importantes en la teoría.
\end{itemize}

\begin{pregunta*}
    ¿Qué tipo de información rescata el functor $\pi$? ¿Posee algún tipo de propiedad universal?
\end{pregunta*}

Consideremos un ejemplo concreto: El $n$-ésimo grupo de homología. Este invariante se codifica a través del functor
\[
    H_n: (f\colon X \to Y) \mapsto (f_*: H_n(X) \to H_n(Y))
\]
En el estudio de la homología singular aprendimos que este functor realmente se factoriza a través de $\pi$:
\[
    H_n = \overline{H_n} \circ \pi
\]
donde
\[
    \overline{H_n}: [f] \in [X,Y] \mapsto f_* \in [H_n(X),H_n(Y)]
\]
Si uno revisa la prueba, consigue observar que para que esto funcione solo necesitamos que $H_n$ mapee equivalencias homotópicas en isomorfismos. 

Esto sugiere una propiedad universal de $\pi$:

\begin{proposicion}[Propiedad universal de $\pi$]
    Sea $\mathcal{C}$ una categoría y $F\colon \mathbf{Top} \to \mathcal{C}$ un functor que mapea equivalencias homotópicas en isomorfismos. Entonces, existe un único functor $\overline{F}\colon \mathbf{HoTop} \to \mathcal{C}$ tal que $F = \overline{F} \circ \pi$.
\end{proposicion}

\begin{nota}
    Este tipo de propiedad universal es fundamental en la teoría de categorías. Esta se encuadra dentro de la construcción de \textbf{categorías localizadas}. De este lenguaje hay varios ejemplos. Por ejemplo, tomar homología en dg-modulos mata los quasi-isomorfismos y por tanto se factoriza a traves de la categoria localizada por la familia de quasi-isomorfismos. En este caso, esta es la categoria derivada. En el presente ejemplo, $\mathbf{HoTop}$ es isomorfa a la localización de $\mathbf{Top}$ por equivalencias homotópicas.
\end{nota}



\end{document}